\section{Experimentelle Bestimmung des Kausalitätsverhaltens}
\label{sec:experimentelle_bestimmung_des_kausalitätsverhaltens}

Wie in Abschnitt \ref{sec:aufbau_des_magnetpendels} erwähnt, dient das Magnetpendel oft zur Erläuterung von schwacher Kausalität, da es bei minimaler Änderung der Position beim Loslassen des Pendels ein komplett anderes Resultat liefert. Das Verhalten der Kausalität bei Veränderung bestimmter Einflussfaktoren wird im Folgenden qualitativ bestimmt.

\subsection{Die Bestimmung der Kausalität}
\label{ssec:die_bestimmung_der_kausalität}

Um die Kausalität zu bestimmen, werden sogenannte Fraktalbilder des Magnetpendels benötigt. Diese stellen eine eingefärbte Draufsicht des Pendels dar, wobei jedem Magneten eine Farbe zugewiesen wird. Jeder Punkt des Bildes ist in der Farbe des Magneten eingefärbt, über dem das Pendel zur Ruhe kommt, wenn der Pendelmagnet am selbigen Punkt losgelassen wird. Dabei gilt für die Geschwindigkeit des Pendelmagneten $\vec{v} = \vec{0}$. Solch ein Fraktalbild wird durch das Abtasten jedes Punktes in einem gewissen Bereich des Pendels erzeugt, wie es im Codebeispiel \ref{code:edl-experimentbeschreibungscode} im Anhang zu sehen ist. Durch zwei verschachtelte \ilc{FOR-LOOPS}, die jeweils eine Zählervariable mitführen, wird das Feld vom Ursprung in x- und y-Richtung je einen halben Meter weit abgetastet. Diese Zählervariablen stellen einen Integer-Wert dar, der von $0$ bis $50$ wandert. Dies entspricht dem Abtasten des ersten Quadranten des hier zweidimensional betrachteten Koordinatensystems bei einer Schrittweite von einem Zentimeter. Es wird hier nur der erste Quadrant abgetastet, da das Modell symmetrisch aufgebaut ist und somit aus einem der Quadranten die Restlichen durch Spiegelung und Farbinvertierung generiert werden können. So kann nach einmaliger Bestätigung der Symmetrie die benötigte Rechenzeit auf ein Viertel reduziert werden.

Für jeden Punkt, also innerhalb der \ilc{FOR-LOOPS}, wird ein Durchlauf der in Abschnitt \ref{sec:erstellung_der_modellbeschreibung} beschriebenen Modellbeschreibung gestartet. Dabei wird aus den Zählervariablen und der Länge der Verbindungsstange des Pendels, die in der Variable \ilc{l} gespeichert ist, die aktuelle Startposition des Pendelmagneten errechnet. Nach dem Ende des Simulationslaufs wird die Modellvariable \ilc{Result} abgefragt, welche, wie in Teilabschnitt \ref{ssec:die_abbruchbedingung} beschrieben, den Endzustand des Pendels enthält. Diese wird dann zusammen mit den x- und y-Koordinaten des Laufs in der Konsole ausgegeben. Danach wird durch die \ilc{FOR-LOOPS} die Position verändert und ein neuer Lauf gestartet.

Aus der so erzeugten Konsolenausgabe kann mithilfe des Programms \cite{extendedPlotter} ein Fraktalbild generiert werden. Dem Magneten $B$ bzw. der Zahl $2$ ist dabei die Farbe blau und dem Magneten $A$, also der Zahl $1$, die Farbe rot zugeordnet. Liefert der Lauf kein Ergebnis, also den Initialwert $0$, so wird der entsprechende Punkt weiß gekennzeichnet.

Die Stärke der Kausalität lässt sich nun anhand der Anzahl und Größe der auf dem Fraktalbild sichtbaren, zusammenhängenden, einfarbigen Felder bestimmen. Besteht ein Fraktalbild aus vielen winzigen Feldern, weißt dies auf schwache Kausalität hin, während große Felder auf starke Kausalität hindeuten.

\subsection{Die Wertfindung}
\label{ssec:die_wertfindung}

Da es für ein Magnetpendel keinerlei Normen gibt, beruhen sämtliche Modellkonstanten wie die Positionen der Fixmagnete oder Größe und Masse des Pendelmagneten auf Schätzungen. Lediglich die Ladungen der elektrostatischen Kugeln $A$ und $B$, sowie des Pendelmagneten bilden dabei eine Ausnahme. Diese sind so gewählt, dass bei jeder im Rahmen der Abtastgenauigkeit erreichbaren Startposition die resultierende Coulombkraft $\vec{F}_{c}$ die auf den Pendelmagneten wirkende Erdanziehungskraft $\vec{F}_{g}$ überwinden kann, sodass lediglich Punkte, die auf der Mittelsenkrechten der Strecke $[AB]$ liegen, kein Resultat liefern, also weiß eingefärbt werden.

\subsection{Die Untersuchung des Kausalitätsverhaltens}
\label{ssec:die_untersuchung_des_kausalitätsverhaltens}

Durch die Durchführung des zuvor beschriebenen Experiments unter verschiedenen Umwelteinflüssen entstanden die in den Abbildungen \ref{fig:Fraktalbilder_bei_veränderter_dichte_des_mediums}, \ref{fig:Fraktalbilder_bei_veränderter_pendelmasse}, \ref{fig:Fraktalbilder_bei_veränderter_magnetstärke} und \ref{fig:Fraktalbilder_bei_veränderter_fallbeschleunigung} dargestellten Fraktalbildreihen. Jede Bildreihe stellt eine Versuchsreihe dar, wobei stets nur ein Wert verändert wird. Dazu wird vor jedem Experimentlauf der Initialwert der entsprechenden Konstante im MDL-Code verändert. Das Teilbild \pa der Fraktalbildreihe zeigt stets das Fraktal mit der schwächsten Kausalität, das Teilbild \pc das Fraktal mit der stärksten Kausalität und das Teilbild \pb den über alle Versuchsreihen gleichbleibenden Vergleichswert. Letzterer ist so gewählt, dass bei der gewählten Auflösung Strukturen im Fraktalbild erkennbar sind. Dadurch bleibt Raum für Veränderungen der Kausalität in beide Richtungen: Entweder hin zu reinem Bildrauschen, also schwacher Kausalität oder komplett einfarbigen Bildhälften und somit vollständiger Determiniertheit.

\paragraph{Dichte des Mediums} Das Fraktal \pa in Abbildung \ref{fig:Fraktalbilder_bei_veränderter_dichte_des_mediums} zeigt klar, wieso das Magnetpendel oft als Beispiel für schwache Kausalität verwendet wird. Bei dieser Versuchsreihe wurde lediglich die Stärke des Strömungswiderstands, oder genauer die Dichte des Mediums, durch das sich der Pendelmagnet bewegt, verändert, sodass das Experiment stets unter erdähnlichen Bedingungen stattfand. In Luft, wo das Pendel üblicherweise anzutreffen ist, sind nur wenige Strukturen erkennbar, die jedoch sehr klein sind. Sie könnten deshalb auch auf Zufall beruhen, da bei der gewählten Abtastrate von einem Zentimeter eine hohe Wahrscheinlichkeit existiert, dass zwischen zwei abgetasteten Punkten mit gleichem Ergebnis ein Punkt mit gegenteiligem Ergebnis unentdeckt bleibt. Je größer aber die zusammenhängende Fläche ist, desto kleiner wird die Wahrscheinlichkeit, dass die Fläche auf das soeben beschriebene Phänomen zurückzuführen ist.

Tauscht man hingegen das Medium gegen eines mit höherer Dichte, so gibt das Pendel seinen chaotischen Charakter schnell auf. Teilbild \pb zeigt den Versuch in Wolframhexafluorid. Dabei handelt es sich um ein Gas mit der Dichte $\rho = 12.7 \frac{kg}{m^3}$, was ca. dem Zehnfachen der Dichte von Luft entspricht. Während sich dort schon größere Strukturen erkennen lassen, welche erste Vorhersagen bei einem mit der Hand durchgeführten Pendellauf ermöglichen würden, ist das Pendel in Wasser komplett determiniert, wie im Fraktal \pc zu erkennen ist. Dies lässt sich dadurch erklären, dass bei der mit $1000 \frac{kg}{m^3}$ überaus hohen Dichte von Wasser der Pendelmagnet einem so starken Strömungswiderstand ausgesetzt ist, dass er nicht genug Geschwindigkeit aufbauen kann, um über einen Fixmagneten hinwegzuschwingen. Demnach kommt das Pendel beim nächstgelegenen Fixmagneten zum Stehen. Bei diesem Versuch müsste allerdings der mit dem Gesetz von Stokes definierte Strömungswiderstand für Bewegungen in Flüssigkeiten verwendet und die Auftriebskraft, die der Pendelmagnet im Wasser erfährt, berücksichtigt werden. Jedoch würde dies in noch viel höheren rücktreibenden Kräften resultieren, sodass der Effekt der Determiniertheit, der bereits maximal ist, nur noch verstärkt würde.

\paragraph{Masse des Pendelmagneten} In der Versuchsreihe aus Abbildung \ref{fig:Fraktalbilder_bei_veränderter_pendelmasse} wurde die Masse des Pendelmagneten variiert. Dies nimmt Einfluss auf dessen Trägheit. Hier zeigt sich, welch großen Einfluss die Ausführung des Pendels auf dessen Kausalität haben kann. Während die Ausführung \pc mit leichtem Pendel stark kausal ist, weißt das Fraktalbild \pa mit schwerem Pendel in großen Teilen starke Ähnlichkeit mit dem Fraktalbild \pa aus Abbildung \ref{fig:Fraktalbilder_bei_veränderter_dichte_des_mediums} auf, obwohl der Versuch des ersteren in Wolframhexafluorid und nicht in Luft durchgeführt wurde.

\paragraph{Stärke der Magnete} Selbiges zeigt sich in den Fraktalbildern aus Abbildung \ref{fig:Fraktalbilder_bei_veränderter_magnetstärke}, bei denen die Ladung der Magnete bzw. elektrostatisch geladenen Kugeln verändert wurde. Der zugehörige Betrag der Ladung ist unter jedem Teilbild angegeben. Dabei bleibt festzuhalten, dass die Determiniertheit mit den magnetischen Anziehungskräften steigt.

\paragraph{Stärke der Fallbeschleunigung} Im Kontrast dazu stehen die Versuche aus Abbildung \ref{fig:Fraktalbilder_bei_veränderter_fallbeschleunigung}. Dort zeigt sich, dass die Kausalität mit zunehmender Fallbeschleunigung von der in Schwerelosigkeit ($0 \frac{m}{s^2}$) bis hin zu der des Jupiters ($24.8 \frac{m}{s^2}$) abnimmt.

\paragraph{Erklärung der Beobachtungen} Relevant für die Kausalität des Pendels ist die Dauer, die der Pendelmagnet benötigt, um so viel Geschwindigkeit abzubauen, dass er von einem Fixmagneten festgehalten werden kann. Hat man zwei Läufe, in denen der Pendelmagnet an nahezu der selben Startposition losgelassen wird, so divergieren die Bahnen dieser Pendelmagnete umso stärker, je länger die Pendel bereits schwingen. Dies ist darauf zurückzuführen, dass die Ablenkung durch die Fixmagnete abhängig von deren Abstand zum Pendelmagneten ist und somit eine Differenz zwischen den Bahnen auch eine Differenz in den Kräften, die die Bahnen beeinflussen, zur Folge hat. Dieses Phänomen ist auch bei normalen Magnetpendeln bei denen Magnete anstelle der geladenen Kugeln verwendet werden zu erwarten, da auch deren Anziehungskräfte zumindest näherungsweise abhängig vom Abstand sind. Besonders stark nimmt die Anziehung dann ab, wenn die Pole der Magneten nicht mehr aufeinandergerichtet sind. Befindet sich der Pendelmagnet also nur minimal neben einem Fixmagneten, so sinkt die Anziehungskraft zwischen den Magneten bereits drastisch. Die auftretenden Differenzen zwischen zwei solchen Bahnen, wie sie zuvor beschrieben wurden, können somit noch stärker ausfallen, als in dem hier implementierten Pendel.

Gegen Ende eines Laufes ist der Pendelmagnet sehr langsam, wobei er meist zwischen den Fixmagneten hin und her schwingt indem er diese halb umrundet, sodass seine Bahn einer Acht gleicht. In dieser Phase kann eine minimale Differenz in Position oder Geschwindigkeit darüber entscheiden, ob der Pendelmagnet seine bisherige Bahn beibehält oder nur noch in einem Kreis um einen der Fixmagnete schwingt. Genauso kann der Pendelmagnet aufgrund solch einer kleinen Differenz aus seiner Bahn um einen der Fixmagnete ausbrechen und wieder zurück in die Bahn um beide Fixmagnete wechseln. So können zwei Läufe, mit nahezu gleichen Anfangsbedingungen ein komplett gegenteiliges Ergebnis liefern. Da die dafür nötige Differenz umso leichter zu erreichen ist, je länger die elektrostatischen Anziehungskräfte wirken konnten, nimmt die Kausalität bzw. Determiniertheit des Magnetpendels mit jeder Sekunde, die es schwingt, ab.

Dies kann anhand der Beobachtungen bestätigt werden. Mit zunehmender Dichte des Mediums steigen die rücktreibenden Kräfte, der Pendelmagnet wird schneller gebremst und schwingt somit nur für eine kürzere Zeit. Steigt die Masse und somit die Trägheit des Pendelmagenten, so benötigt es bei gleicher Kraft mehr Zeit diesen zu bremsen. Mit der Ladung der elektrostatischen Kugeln steigt die Kraft, mit der der Pendelmagnet in der Kreisbahn um einen der Fixmagnete gehalten werden kann. Dies führt dazu, dass dieser nicht so stark abgebremst werden muss, um einen vollständig determinierten Zustand zu erreichen. Schließlich führt eine erhöhte Fallbeschleunigung $g$ laut der Formel $E_{pot} = m \cdot g \cdot \Delta h$ zu einer höheren potentiellen Energie, die der Pendelmagnet zu Beginn des Laufs bei konstanter Höhendifferenz $\Delta h$ aufweist. Diese wird im Zuge jedes Pendelschwunges in kinetische Energie also Geschwindigkeit umgewandelt. Da diese somit ebenso steigt, wird mehr Zeit benötigt, um den Pendelmagneten zu bremsen.\\

\textbf{Die Dauer des Pendelvorgangs ist somit ausschlaggebend für die Stärke der Kausalität eines Magnetpendels. Je länger ein Pendel aufgrund seines Aufbaus und der gegebenen Umwelteinflüsse durchschnittlich schwingt, desto chaotischer ist es.}